\documentclass[a4paper,12pt]{report}

\usepackage{alltt, fancyvrb, url}
\usepackage{graphicx}
\usepackage[utf8]{inputenc}
\usepackage{float}
\usepackage{hyperref}
\usepackage{adjustbox}
\usepackage{siunitx}
\usepackage{tabularx}
\usepackage{enumitem}
\sisetup{group-separator={\text{\space}}}

% Questo commentalo se vuoi scrivere in inglese.
\usepackage[italian]{babel}

\usepackage[italian]{cleveref}

\title{Relazione per\\``Basi di dati''}

\author{Nicolò Guerra \and
Filippo Casadei}

\begin{document}

\maketitle

\tableofcontents

\chapter{Analisi dei requisiti}

Si vuole realizzare un database per la gestione di sistemi ospedalieri. Il sistema dovrà immagazzinare dati relativi a diverse ASL,
agli ospedali ad esse associate, pazienti, medici. Dovrà registrare inoltre referti relativi a visite, interventi e appuntamenti.

\section{Intervista}
Una prima descrizione delle richieste è la seguente:
\paragraph{}
Si vuole tenere traccia dei referti prodotti nei vari ospedali delle varie ASL della regione. Un referto può essere prodotto da un
intervento o da una visita/esame. Un referto è prodotto da un medico in un ospedale, ed è riferito ad un paziente. Dei pazienti si
vuole memorizzare nome, cognome, codice fiscale, data di nascita, numero di telefono, ASL di appartenenza e referti associati.
Di un impiegato si vuole memorizzare nome, cognome, codice fiscale e ruolo. Di un medico si vuole inoltre tenere traccia dello
storico degli interventi.
Un referto viene prodotto in una specifica data e nel caso sia di un intervento, si vuole sapere la procedura, l'esito, la durata e i medici
coinvolti. Nel caso sia di una visita invece si vuole avere una breve descrizione di quest'ultima e l'eventuale terapia prescritta.
Si vogliono memorizzare inoltre i seguenti dati per gli ospedali: nome, indirizzo, ASL di appartenenza, posti disponibili e persone
che lavorano in una struttura, e se questo dispone di un pronto soccorso.
Un ospedale si compone inoltre di più unità operative, ognuna caratterizzata da un nome e dal personale che vi lavora.
Ogni ospedale ha delle attrezzature, identificate da un codice di inventario e di cui si vuole memorizzare il nome e la data dell'ultima
manutenzione.
Si vogliono infine memorizzare degli appuntamenti, che possono essere ad esempio prenotazioni per interventi e coinvolgono uno o più
pazienti e uno o più medici in una certa sala dell'ospedale ad una precisa data e ora. Un medico o un paziente non possono avere più
appuntamenti nello stesso momento. Una sala non può essere occupata da 2 appuntamenti contemporaneamente.
\section{Concetti principali da modellare}
\begin{itemize}
  \item ASL: Azienda sanitaria locale, gestisce la sanità di una singola zona di competenza, solitamente provinciale
  \item Ospedale: Struttura in cui avvengono visite e interventi dei pazienti
  \item Referto: Documento prodotto da un medico come resoconto di una visita o un intervento
  \item Paziente: Persona sottoposta a trattamenti ospedalieri
  \item Medico: Persona che lavora in un ospedale (o più di uno) e effettua trattamenti
  \item Impiegato: Persona che lavora in un ospedale ma non effettua trattamenti e non può firmare referti
  \item Unità operativa (U.O.): Reparto dell'ospedale specializzato in determinati tipi di trattamenti
  \item Attrezzatura: Macchinario utilizzato per particolari interventi e/o visite
  \item Appuntamento: Programmazione di visita o intervento
  \item Visita: Consulenza con un medico riguardante lo stato di salute del paziente
  \item Intervento: Operazione ad un paziente da parte di uno o più medici
  \item Sala: Stanza in cui avvengono visite e interventi
\end{itemize}
\section{Riscrittura dell'intervista corretta}
Di un referto si vuole memorizzare la data di emissione, il paziente, i medici coinvolti e il tipo di referto, oltre ad una breve descrizione.
Se è stato prodotto per un intervento si memorizzano anche procedura, esito e durata, se invece è per una visita la terapia prescritta.

Di una persona si vogliono memorizzare nome, cognome, codice fiscale, numero di telefono.
Di un paziente inoltre si vogliono memorizzare data di nascita, ASL di appartenenza e i referti associati.
Di un impiegato si vuole memorizzare il ruolo all'interno della struttura.
Di un medico si vuole memorizzare lo storico degli interventi.

Di un ospedale si vogliono memorizzare nome, indirizzo, ASL di appartenenza, posti disponibili (inteso come capienza totale in numero di pazienti),
persone che vi lavorano e il loro numero. Si vuole inoltre sapere se l'ospedale è munito di pronto soccorso.

Di una unità operativa si vogliono memorizzare il nome, l'ospedale a cui appartiene, il personale che vi lavora, la capienza e i posti liberi.

Di ogni attrezzatura si vogliono memorizzare il codice di inventario, l'ospedale di cui fanno parte, il nome e la data dell'ultima manutenzione.

Di ogni appuntamento si vuole memorizzare la data e l'ora, la durata stimata, il tipo, la sala, i pazienti e i medici coinvolti. 
Un paziente e/o un medico non possono avere più appuntamenti nello stesso lasso di tempo, inoltre una sala non può essere occupata da più 
appuntamenti allo stesso tempo.
\section{Principali azioni richieste}
\begin{itemize}
  \item Aggiungere un nuovo ospedale
  \item Aggiungere una nuova unità operativa
  \item Rimuovere un'unità operativa da un ospedale
  \item Aggiungere un nuovo paziente
  \item Aggiungere un nuovo impiegato/medico
  \item Fissare un appuntamento
  \item Cancellare un appuntamento
  \item Aggiungere un nuovo referto
  \item Ricercare i referti per medico o per paziente
  \item Aggiungere nuova attrezzatura
  \item Rimuovere attrezzatura
  \item Aggiornare la data di manutenzione di un'attrezzatura
  \item Ricercare ospedali con determinate unità operative
  \item Ricerca di ospedali appartenenti ad un'ASL specifica
  \item Aggiungere una sala ad un ospedale
  \item Rimuovere una sala da un ospedale
  \item Ricercare ospedali con posti liberi in una determinata unità operativa
  \item Aggiungere pazienti in cura presso un ospedale
  \item Rimuovere un paziente in cura presso un ospedale
\end{itemize}

\chapter{Progettazione concettuale}
\begin{figure}[H]
	\centering{}
	\includegraphics[width=\textwidth]{img/scheletro.png}
	\caption{Schema scheletro del problema.}
	\label{img:scheletro}
\end{figure}
Dopo l'esame del dominio del problema risultano alcune considerazioni da fare, che implicano diversi raffinamenti possibili:

Va considerata la presenza di diverse categorie di lavoratori all'interno di un ospedale e solo alcuni di questi devono essere in grado di rilasciare
referti ai pazienti. Per questi motivi si è deciso di dividere l'entità dipendenti dello schema scheletro in amministrativi e personale sanitario, che assieme a pazienti
risultano essere sottocategorie dell'entità persona, identificabili mediante codice fiscale.

Inoltre unicamente gli amministrativi sono collegati direttamente all'ospedale, mentre il personale sanitario si relaziona ad un'unità operativa, quest'ultima associata 
ad un ospedale.

Gli appuntamenti vengono modellati come entità perché devono essere identificati da data, ora e luogo in cui avvengono, attributi propri dell'appuntamento 
stesso. Non risultano però modellabili alcuni requisiti. Essendo gli appuntamenti protratti nel tempo non si può modellare nello schema E-R il vincolo di 
non poter avere 2 appuntamenti sovrapposti. Diversamente da come descritto sopra un appuntamento non è direttamente connesso ad un referto, ma esprime
unicamente il concetto di incontro tra pazienti e medici.

Anche visite e interventi sono estensioni della generica entità referto, ne rappresentano infatti il tipo, e i dati specifici per l'uno o per l'altro caso. Inoltre per
poter ricostruire lo storico degli interventi di un medico ogni referto è anche associato al medico che lo ha prodotto.

È stata anche aggiunta la relazione di registrazione di un paziente presso una determinata ASL.

Lo schema E/R definitivo risulta quindi essere il seguente:
\begin{figure}[p]
  \begin{adjustbox}{addcode={\begin{minipage}{\width}}{\caption{%
    Schema E/R del problema.
    }\end{minipage}},rotate=270,center}
    \includegraphics[height=0.93\textwidth]{img/er_final.png}
    \label{img:er}
  \end{adjustbox}
\end{figure}

\chapter{Progettazione logica}
\section{Stima del volume dei dati}
Si stimano nella tabella seguente il volume dei dati previsto nel database:
\subsection{Entità}
\begin{center}
  \begin{tabular}{ c | S[table-format=8.0, table-space-text-pre=(, table-space-text-post=)] }
    Nome & Quantità \\
    \hline
    Amministrativi & 40000 \\
    Appuntamento & 12000000 \\
    ASL & 100 \\
    Attrezzatura & 15000 \\
    Intervento & 200000 \\
    Ospedale & 1000 \\
    Paziente & 50000000 \\
    Persona & 50000000 \\
    Personale sanitario & 500000 \\
    Sala & 30000 \\
    Unità operativa & 10000\\
    Visita & 7000000 \\
  \end{tabular}
\end{center}
 
Va considerato che essendo Personale sanitario, Amministrativi e Paziente gerarchie di persona il loro volume è 
sovrapposto (la copertura è totale ma non esclusiva). 
Anche Intervento e Visita sono gerarchie di Referto, ma essendo questa volta una copertura totale e disgiunta Referto
è stato omesso perché ricavabile come somma di queste 2.
I volumi di Intervento e Visita sono inoltre calcolati su un anno di operatività. Questo perché sono dati che crescono molto rapidamente 
e si rende necessario un sistema di archiviazione a lungo termine per evitare di sovraccaricare il sistema e causare inefficienze.
Anche gli appuntamenti soffrono dello stesso problema, per cui vengono eliminati una volta effettuati.

\subsection{Relazioni}
\begin{center}
  \begin{tabular}{ c | S[table-format=8.0, table-space-text-pre=(, table-space-text-post=)] }
    Nome & Quantità \\
    \hline
    Appartenenza & 1000 \\
    Associazione & 7200000 \\
    Coinvolgimento & 8800000 \\
    Composizione & 10000 \\
    Cura & 6000000 \\
    Emissione & 7200000 \\
    Impiegati & 40000 \\
    Lavora & 700000 \\
    Possiede & 15000 \\
    Prenota & 12000000 \\
    Presenzia & 15000000 \\
    Registrazione & 45000000 \\
    Si svolge & 12000000 \\
    Struttura & 30000 \\
  \end{tabular}
\end{center}

Si consideri che come nella sezione precedente anche qui alcuni dati sono stati stimati su un anno di operatività. 
Cura è una relazione che cambia rapidamente nel tempo e per evitare il sovraccarico del sistema va archiviata a intervalli regolari.

\subsection{Principali operazioni e loro frequenza}
Segue ora una tabella in cui vengono riportate le principali operazioni da effettuare sul database e la loro frequenza stimata.
\begin{center}
  \begin{tabular}{ p{269pt} | l }
    Operazione & Frequenza \\
    \hline
    \hline
    Aggiungere un nuovo ospedale & 3 all'anno \\
    Aggiungere una nuova unità operativa & 25 all'anno \\
    Rimuovere un'unità operativa da un ospedale & 10 all'anno \\
    Aggiungere un nuovo paziente & 1000 al giorno \\
    Aggiungere un nuovo impiegato/medico & 2000 al mese \\
    Fissare un appuntamento & 30000 al giorno \\
    Cancellare un appuntamento & 3000 al giorno \\
    Aggiungere un nuovo referto & 10000 al giorno \\
    Ricercare i referti per medico o per paziente & 7000 al giorno \\
    Aggiungere nuova attrezzatura & 40 all'anno \\
    Rimuovere attrezzatura & 30 all'anno \\
    Aggiornare la data di manutenzione di un'attrezzatura & 22500 all'anno \\
    Ricercare ospedali con determinate unità operative & 750 al giorno \\
    Ricerca di ospedali appartenenti ad un'ASL specifica & 15000 al giorno \\
    Aggiungere una sala ad un ospedale & 10 all'anno \\
    Rimuovere una sala da un ospedale & 5 all'anno \\
    Ricercare ospedali con posti liberi in una determinata unità operativa & 200 al giorno \\
    Aggiungere pazienti in cura presso un ospedale & 12500 al giorno \\
    Rimuovere un paziente in cura presso un ospedale & 12500 al giorno \\
  \end{tabular}
\end{center}
\subsection{Dettaglio delle operazioni e tabelle degli accessi}
Sono ora descritte le operazioni nel dettaglio con la stima del loro costo e le tabelle degli accessi. A questo fine gli 
accessi in scrittura vengono considerati di costo doppio rispetto a quelli in lettura.

Per operazioni particolarmente complesse vengono inoltre riportati gli schemi di navigazione.

\subsubsection*{1 - Aggiungere un nuovo ospedale}
Conoscendo già il codice dell'ASL a cui l'ospedale da aggiungere dovrà appartenere questa operazione richiede la scrittura dell'entità ospedale 
e della sua relazione con ASL (appartenenza).
\vspace{6pt}
\newline
\begin{tabularx}{\textwidth}{ 
  | >{\centering\arraybackslash}X 
  | >{\centering\arraybackslash}X 
  | >{\centering\arraybackslash}X 
  | >{\centering\arraybackslash}X |}
  \hline
  Soggetto & E/R & Accessi & R/W \\
  \hline
  \hline
  Ospedale & E & 1 & W \\ 
  \hline
  Appartenenza & R & 1 & W \\
  \hline
\end{tabularx}
\vspace{3pt}\newline
Costo operazione: 1W + 1W = 4 \newline Costo totale: 4 * 3 (all'anno) = 12 all'anno

\subsubsection*{2 - Aggiungere una nuova unità operativa}
Conoscendo già il codice dell'ospedale a cui l'unità operativa verrà aggiunta questa operazione richiede la scrittura dell'entità unità operativa 
e della sua relazione con ospedale (composizione).
\vspace{6pt}
\newline
\begin{tabularx}{\textwidth}{ 
  | >{\centering\arraybackslash}X 
  | >{\centering\arraybackslash}X 
  | >{\centering\arraybackslash}X 
  | >{\centering\arraybackslash}X |}
  \hline
  Soggetto & E/R & Accessi & R/W \\
  \hline
  \hline
  Unità operativa & E & 1 & W \\ 
  \hline
  Composizione & R & 1 & W \\
  \hline
\end{tabularx}
\vspace{3pt}\newline
Costo operazione: 1W + 1W = 4 \newline Costo totale: 4 * 25 (all'anno) = 100 all'anno

\subsubsection*{3 - Rimuovere un'unità operativa da un ospedale}
Rimuovere un unità operativa significa rimuovere anche la sua associazione con l'ospedale a cui apparteneva e l'associazione del personale sanitario 
attualmente assegnato a quell'unità, in media 40 per unità.
\vspace{6pt}
\newline
\begin{tabularx}{\textwidth}{ 
  | >{\centering\arraybackslash}X 
  | >{\centering\arraybackslash}X 
  | >{\centering\arraybackslash}X 
  | >{\centering\arraybackslash}X |}
  \hline
  Soggetto & E/R & Accessi & R/W \\
  \hline
  \hline
  Unità operativa & E & 1 & W \\ 
  \hline
  Composizione & R & 1 & W \\
  \hline
  Lavora & R & 40 & W \\
  \hline
\end{tabularx}
\vspace{3pt}\newline
Costo operazione: 1W + 1W + 40W = 84 \newline Costo totale: 84 * 10 (all'anno) = 840 all'anno

\subsubsection*{4 - Aggiungere un nuovo paziente}
Aggiungere un paziente richiede la scrittura di una tupla nell'entità \emph{Paziente}. Viene considerata anche la scrittura della relazione \emph{Registrazione}
nonostante non sia obbligatoria perché la quasi totalità dei pazienti è registrata presso un'ASL.
\vspace{6pt}
\newline
\begin{tabularx}{\textwidth}{ 
  | >{\centering\arraybackslash}X 
  | >{\centering\arraybackslash}X 
  | >{\centering\arraybackslash}X 
  | >{\centering\arraybackslash}X |}
  \hline
  Soggetto & E/R & Accessi & R/W \\
  \hline
  \hline
  Paziente & E & 1 & W \\ 
  \hline
  Registrazione & R & 1 & W \\
  \hline
\end{tabularx}
\vspace{3pt}\newline
Costo operazione: 1W + 1W = 4 \newline Costo totale: 4 * 1000 (al giorno) = 4000 al giorno

\subsubsection*{5 - Aggiungere un nuovo impiegato/medico}
Aggiungere un medico, assumendo che le unità operative vengano assegnate in un secondo momento, richiede una nuova tupla nell'entità \emph{Personale sanitario}.
\vspace{6pt}
\newline
\begin{tabularx}{\textwidth}{ 
  | >{\centering\arraybackslash}X 
  | >{\centering\arraybackslash}X 
  | >{\centering\arraybackslash}X 
  | >{\centering\arraybackslash}X |}
  \hline
  Soggetto & E/R & Accessi & R/W \\
  \hline
  \hline
  Personale sanitario & E & 1 & W \\ 
  \hline
\end{tabularx}
\vspace{3pt}\newline
Costo operazione: 1W = 2 \newline Costo totale: 2 * 1200 (al mese) = 2400 al mese
\vspace{3pt}
\newline
Aggiungere un impiegato invece richiede la scrittura di una tupla in \emph{Personale amministrativo}, e della relazione \emph{Impiegati}.
\vspace{6pt}
\newline
\begin{tabularx}{\textwidth}{ 
  | >{\centering\arraybackslash}X 
  | >{\centering\arraybackslash}X 
  | >{\centering\arraybackslash}X 
  | >{\centering\arraybackslash}X |}
  \hline
  Soggetto & E/R & Accessi & R/W \\
  \hline
  \hline
  Amministrativi & E & 1 & W \\
  \hline
  Impiegati & R & 1 & W \\
  \hline
\end{tabularx}
\vspace{3pt}\newline
Costo operazione: 2W = 4 \newline Costo totale: 4 * 800 (al mese) = 3200 al mese

\subsubsection*{6 - Fissare un appuntamento}
Fissare un appuntamento richiede di verificare che il medico non sia occupato in un altro appuntamento e che la sala sia libera in un dato orario.
Per controllare che il medico non sia occupato è necessario leggere gli appuntamenti di un medico in una certa data, quindi un numero pari alla media
di appuntamenti giornalieri di un medico dalla relazione \emph{Presenzia}, dopodiche leggere dall'entità \emph{Appuntamento} quanti di questi hanno orario e durata
tali da sovrapporsi a quello che stiamo inserendo. Dobbiamo poi leggere gli appuntamenti che si svolgono in un giorno in una stanza e controllare anche qui se ci sono
sovrapposizioni.
\vspace{6pt}
\newline
\begin{tabularx}{\textwidth}{ 
  | >{\centering\arraybackslash}X 
  | >{\centering\arraybackslash}X 
  | >{\centering\arraybackslash}X 
  | >{\centering\arraybackslash}X |}
  \hline
  Soggetto & E/R & Accessi & R/W \\
  \hline
  Presenzia & R & 4 & R \\
  \hline
  Si svolge & R & 8 & R \\
  \hline
  Appuntamento & E & 1 & W \\
  \hline
  Si svolge & R & 1 & W \\
  \hline
  Prenota & R & 1 & W \\
  \hline
  Presenzia & R & 1 & W \\
  \hline
\end{tabularx}
\vspace{3pt}\newline
Costo operazione: 4R + 8R + 1W + 1W + 1W + 1W = 20 \newline Costo totale: 20 * 30000 (al giorno) = 600000 al giorno
\begin{figure}[H]
	\centering{}
	\includegraphics[width=\textwidth]{img/nav_appuntamento.png}
	\caption{Schema di navigazione.}
	\label{img:nav_appuntamento}
\end{figure}

\subsubsection*{7 - Cancellare un appuntamento}
Cancellare un appuntamento richiede di eliminare la relativa tupla dall'entità appuntamento, e le 3 relazioni che la caratterizzano. In condizioni
normali una visita è presieduta da un solo medico, quindi si considera questo caso ai fini del calcolo dei costi.
\vspace{6pt}
\newline
\begin{tabularx}{\textwidth}{ 
  | >{\centering\arraybackslash}X 
  | >{\centering\arraybackslash}X 
  | >{\centering\arraybackslash}X 
  | >{\centering\arraybackslash}X |}
  \hline
  Soggetto & E/R & Accessi & R/W \\
  \hline
  Appuntamento & E & 1 & W \\
  \hline
  Prenota & R & 1 & W \\
  \hline
  Si svolge & R & 1 & W \\
  \hline
  Presenzia & R & 1 & W \\
  \hline
\end{tabularx}
\vspace{3pt}\newline
Costo operazione: 1W + 1W + 1W + 1W = 8 \newline Costo totale: 8 * 3000 (al giorno) = 24000 al giorno

\subsubsection*{8 - Aggiungere un nuovo referto}
\emph{Referto} è un'entità che ha come sotto entità nella gerarchia \emph{Intervento} e \emph{Visita}, essendo però una copertura totale ed esclusiva e non avendo le 
sotto entità altre relazioni, consideriamo solo \emph{Referto} nel calcolo dei costi per l'inserimento. La media delle associazioni \emph{Coinvolgimento} tra \emph{Personale sanitario}
e \emph{Referto} è ottenuta considerando che la maggioranza delle visite è tra un solo medico e il paziente, mentre quasi tutti gli interventi richiedono più personale.
\vspace{6pt}
\newline
\begin{tabularx}{\textwidth}{ 
  | >{\centering\arraybackslash}X 
  | >{\centering\arraybackslash}X 
  | >{\centering\arraybackslash}X 
  | >{\centering\arraybackslash}X |}
  \hline
  Soggetto & E/R & Accessi & R/W \\
  \hline
  Referto & E & 1 & W \\
  \hline
  Associazione & R & 1 & W \\
  \hline
  Emissione & R & 1 & W \\
  \hline
  Coinvolgimento & R & 2 & W \\
  \hline
\end{tabularx}
\vspace{3pt}\newline
Costo operazione: 1W + 1W + 1W + 2W = 10 \newline Costo totale: 10 * 10000 (al giorno) = 100000 al giorno

\subsubsection*{9 - Ricercare i referti per medico o per paziente}
Ricercare i referti per paziente assumendo di conoscere già il codice fiscale del paziente richiede di leggere tutte le tuple legate a quel paziente in \emph{Associazione},
e quindi tutti i referti associati. Si considerano 2 referti per ogni persona come media. Dato che è necessario sapere dove un referto è stato emesso e chi è coinvolto si
considerano anche le letture di queste relazioni.
\vspace{6pt}
\newline
\begin{tabularx}{\textwidth}{ 
  | >{\centering\arraybackslash}X 
  | >{\centering\arraybackslash}X 
  | >{\centering\arraybackslash}X 
  | >{\centering\arraybackslash}X |}
  \hline
  Soggetto & E/R & Accessi & R/W \\
  \hline
  Associazione & R & 2 & R \\
  \hline
  Referto & E & 2 & R \\
  \hline
  Emissione & R & 2 & R \\
  \hline
  Coinvolgimento & R & 2 & R \\
  \hline
\end{tabularx}
\vspace{3pt}\newline
Costo operazione: 2R + 2R + 2R + 2R = 8 \newline Costo totale: 8 * 6000 (al giorno) = 48000 al giorno
\vspace{3pt}
\begin{figure}[H]
	\centering{}
	\includegraphics[width=\textwidth]{img/nav_ref_paziente.png}
	\caption{Schema di navigazione.}
	\label{img:nav_ref_paziente}
\end{figure}
\noindent Ricercare i referti associati ad un medico richiede invece la lettura in coinvolgimento di tutte le associazioni \emph{Coinvolgimento} tra quel medico e un referto. 
Dopodiche si procede a ricostruire le relazioni di \emph{Referto} come sopra.
\vspace{6pt}
\newline
\begin{tabularx}{\textwidth}{ 
  | >{\centering\arraybackslash}X 
  | >{\centering\arraybackslash}X 
  | >{\centering\arraybackslash}X 
  | >{\centering\arraybackslash}X |}
  \hline
  Soggetto & E/R & Accessi & R/W \\
  \hline
  \hline
  Coinvolgimento & R & 18 & R \\
  \hline
  Referto & E & 18 & R \\
  \hline
  Associazione & R & 18 & R \\
  \hline
  Emissione & R & 18 & R \\
  \hline
\end{tabularx}
\vspace{3pt}\newline
Costo operazione: 18R + 18R + 18R + 18R = 72 \newline Costo totale: 72 * 1000 (al giorno) = 72000 al giorno
\vspace{3pt}
\begin{figure}[H]
	\centering{}
	\includegraphics[width=\textwidth]{img/nav_ref_medico.png}
	\caption{Schema di navigazione.}
	\label{img:nav_ref_medico}
\end{figure}
\subsubsection*{10 - Aggiungere nuova attrezzatura}
L'aggiunta di nuova attrezzatura richiede di aggiungere una tupla all'entità \emph{Attrezzatura}, e di registrare la sua associazione \emph{Possiede} con l'ospedale.
\vspace{6pt}
\newline
\begin{tabularx}{\textwidth}{ 
  | >{\centering\arraybackslash}X 
  | >{\centering\arraybackslash}X 
  | >{\centering\arraybackslash}X 
  | >{\centering\arraybackslash}X |}
  \hline
  Soggetto & E/R & Accessi & R/W \\
  \hline
  Attrezzatura & E & 1 & W \\
  \hline
  Possiede & R & 1 & W \\
  \hline
\end{tabularx}
\vspace{3pt}\newline
Costo operazione: 1W + 1W = 4 \newline Costo totale: 4 * 40 (all'anno) = 160 all'anno

\subsubsection*{11 - Rimuovere attrezzatura}
La rimozione dell'attrezzatura richiede di rimuovere una tupla dall'entità \emph{Attrezzatura}, e di eliminare la sua associazione \emph{Possiede} con l'ospedale.
\vspace{6pt}
\newline
\begin{tabularx}{\textwidth}{ 
  | >{\centering\arraybackslash}X 
  | >{\centering\arraybackslash}X 
  | >{\centering\arraybackslash}X 
  | >{\centering\arraybackslash}X |}
  \hline
  Soggetto & E/R & Accessi & R/W \\
  \hline
  Attrezzatura & E & 1 & W \\
  \hline
  Possiede & R & 1 & W \\
  \hline
\end{tabularx}
\vspace{3pt}\newline
Costo operazione: 1W + 1W = 4 \newline Costo totale: 4 * 30 (all'anno) = 120 all'anno

\subsubsection*{12 - Aggiornare la data di manutenzione di un’attrezzatura}
L'aggiornamento della data di ultima manutenzione dell'attrezzatura richiede di aggiornare una tupla nell'entità \emph{Attrezzatura}.
\vspace{6pt}
\newline
\begin{tabularx}{\textwidth}{ 
  | >{\centering\arraybackslash}X 
  | >{\centering\arraybackslash}X 
  | >{\centering\arraybackslash}X 
  | >{\centering\arraybackslash}X |}
  \hline
  Soggetto & E/R & Accessi & R/W \\
  \hline
  Attrezzatura & E & 1 & W \\
  \hline
\end{tabularx}
\vspace{3pt}\newline
Costo operazione: 1W = 2 \newline Costo totale: 2 * 22500 (all'anno) = 45000 all'anno

\subsubsection*{13 - Ricercare ospedali con determinate unità operative}
Ricercare ospedali con determinate unità operative richiede di leggere tutte le tuple di \emph{Composizione} con unità operativa corrispondente a quella richiesta, e da quelle
leggere le tuple dell'entità \emph{Ospedale} associate.
\vspace{6pt}
\newline
\begin{tabularx}{\textwidth}{ 
  | >{\centering\arraybackslash}X 
  | >{\centering\arraybackslash}X 
  | >{\centering\arraybackslash}X 
  | >{\centering\arraybackslash}X |}
  \hline
  Soggetto & E/R & Accessi & R/W \\
  \hline
  Composizione & R & 140 & R \\
  \hline
  Ospedale & E & 140 & R \\
  \hline
\end{tabularx}
\vspace{3pt}\newline
Costo operazione: 140R + 140R = 280 \newline Costo totale: 280 * 750 (al giorno) = 210000 al giorno

\subsubsection*{14 - Ricerca di ospedali appartenenti ad un’ASL specifica}
La ricerca di ospedali appartenenti ad una specifica ASL assumendo di conoscerne il codice richiede un numero di letture della relazione \emph{Appartenenza} pari
al numero medio di ospedali per ASL.
\vspace{6pt}
\newline
\begin{tabularx}{\textwidth}{ 
  | >{\centering\arraybackslash}X 
  | >{\centering\arraybackslash}X 
  | >{\centering\arraybackslash}X 
  | >{\centering\arraybackslash}X |}
  \hline
  Soggetto & E/R & Accessi & R/W \\
  \hline
  Appartenenza & R & 10 & R \\
  \hline
  Ospedale & E & 10 & R \\
  \hline
\end{tabularx}
\vspace{3pt}\newline
Costo operazione: 10R + 10R = 20 \newline Costo totale: 20 * 15000 (al giorno) = 300000 al giorno

\subsubsection*{15 - Aggiungere una sala ad un ospedale}
L'aggiunta di una sala ad un ospedale richiede l'aggiunta di una tupla all'entità \emph{Sala} e della sua associazione \emph{Struttura} con ospedale.
\vspace{6pt}
\newline
\begin{tabularx}{\textwidth}{ 
  | >{\centering\arraybackslash}X 
  | >{\centering\arraybackslash}X 
  | >{\centering\arraybackslash}X 
  | >{\centering\arraybackslash}X |}
  \hline
  Soggetto & E/R & Accessi & R/W \\
  \hline
  Sala & E & 1 & W \\
  \hline
  Struttura & R & 1 & W \\
  \hline
\end{tabularx}
\vspace{3pt}\newline
Costo operazione: 1W + 1W = 4 \newline Costo totale: 4 * 10 (all'anno) = 40 all'anno

\subsubsection*{16 - Rimuovere una sala da un ospedale}
La rimozione di una sala da un ospedale richiede la rimozione di una tupla dall'entità \emph{Sala} e della sua associazione \emph{Struttura} con ospedale.
Non viene considerata la rimozione degli appuntamenti associati a quella sala perché è opportuno non rimuoverla qualora ci siano degli appuntamenti fissati ma prima spostare
questi ultimi in altre sale.
\vspace{6pt}
\newline
\begin{tabularx}{\textwidth}{ 
  | >{\centering\arraybackslash}X 
  | >{\centering\arraybackslash}X 
  | >{\centering\arraybackslash}X 
  | >{\centering\arraybackslash}X |}
  \hline
  Soggetto & E/R & Accessi & R/W \\
  \hline
  Sala & E & 1 & W \\
  \hline
  Struttura & R & 1 & W \\
  \hline
\end{tabularx}
\vspace{3pt}\newline
Costo operazione: 1W + 1W = 4 \newline Costo totale: 4 * 5 (all'anno) = 20 all'anno

\subsubsection*{17 - Ricercare ospedali con posti liberi in una determinata unità operativa}
Inanzitutto è necessario ricercare le unità operative corrispondenti a quella richiesta e i cui posti occupati siano inferiori alla capienza.
Infine dall'unità bisogna risalire all'ospedale attraverso la relazione \emph{Composizione}. Di tutte le unità operative vengono considerate corrispondenti
a quella cercata 1 su 70, e di queste con posti liberi i \( \frac{3}{4} \).
\vspace{6pt}
\newline
\begin{tabularx}{\textwidth}{ 
  | >{\centering\arraybackslash}X 
  | >{\centering\arraybackslash}X 
  | >{\centering\arraybackslash}X 
  | >{\centering\arraybackslash}X |}
  \hline
  Soggetto & E/R & Accessi & R/W \\
  \hline
  Unità operative & E & 105 & R \\
  \hline
  Ospedale & E & 105 & R \\
  \hline
\end{tabularx}
\vspace{3pt}\newline
Costo operazione: 105R + 105R = 210 \newline Costo totale: 210 * 200 (al giorno) = 42000 al giorno

\subsubsection*{18 - Aggiungere pazienti in cura presso un ospedale}
Bisogna controllare che un paziente non sia già in cura presso una struttura, leggendo le tuple di cura associate a un paziente e controllare se ne esistono
senza l'attributo data di uscita. Diamo per scontato che si cerchi di aggiungere un paziente ad un unità operativa che non è piena. Per aggiungere un paziente 
in cura presso un ospedale è sufficiente aggiungere una tupla alla relazione \emph{Cura} e aggiornare i posti occupati nell'unità operativa.
\vspace{6pt}
\newline
\begin{tabularx}{\textwidth}{ 
  | >{\centering\arraybackslash}X 
  | >{\centering\arraybackslash}X 
  | >{\centering\arraybackslash}X 
  | >{\centering\arraybackslash}X |}
  \hline
  Soggetto & E/R & Accessi & R/W \\
  \hline
  Cura & R & 10 & R \\
  \hline
  Cura & R & 1 & W \\
  \hline
  Unità operativa & E & 1 & W \\
  \hline 
\end{tabularx}
\vspace{3pt}\newline
Costo operazione: 10R + 1W + 1W = 14 \newline Costo totale: 14 * 12500 (al giorno) = 175000 al giorno

\subsubsection*{19 - Rimuovere un paziente in cura}
Per rimuovere un paziente in cura presso un ospedale è sufficiente aggiornare la tupla nella relazione \emph{Cura} con la data di uscita del paziente e aggiornare 
i posti occupati nell'unità operativa.
\vspace{6pt}
\newline
\begin{tabularx}{\textwidth}{ 
  | >{\centering\arraybackslash}X 
  | >{\centering\arraybackslash}X 
  | >{\centering\arraybackslash}X 
  | >{\centering\arraybackslash}X |}
  \hline
  Soggetto & E/R & Accessi & R/W \\
  \hline
  Cura & R & 1 & W \\
  \hline
  Unità operativa & E & 1 & W \\
  \hline 
\end{tabularx}
\vspace{3pt}\newline
Costo operazione: 1W + 1W = 4 \newline Costo totale: 4 * 12500 (al giorno) = 50000 al giorno

\section{Raffinamento dello schema}
\subsection{Analisi delle ridondanze}
\subsubsection*{Capienza unità operative}
L'attributo \emph{Posti\_occupati} dell'entità \emph{Unità operativa} è ridondante, in quanto sarebbe deducibile dalla relazione \emph{Cura} contando le tuple corrispondenti
a quell'unità operativa che non contengono l'attributo \emph{Data\_uscita}.

Segue l'analisi del costo delle operazioni interessate:

\subsubsection*{17 - Ricercare ospedali con posti liberi in una determinata unità operativa}
\textbf{Con ridondanza:}
\vspace{6pt}
\newline
\begin{tabularx}{\textwidth}{ 
  | >{\centering\arraybackslash}X 
  | >{\centering\arraybackslash}X 
  | >{\centering\arraybackslash}X 
  | >{\centering\arraybackslash}X |}
  \hline
  Soggetto & E/R & Accessi & R/W \\
  \hline
  Unità operativa & E & 105 & R \\
  \hline
  Ospedale & E & 105 & R \\
  \hline
\end{tabularx}
\vspace{3pt}\newline
Costo operazione: 105R + 105R = 210 \newline Costo totale: 210 * 200 (al giorno) = 42000 al giorno
\vspace{6pt}
\newline
\textbf{Senza ridondanza:}
\vspace{6pt}
\newline
\begin{tabularx}{\textwidth}{ 
  | >{\centering\arraybackslash}X 
  | >{\centering\arraybackslash}X 
  | >{\centering\arraybackslash}X 
  | >{\centering\arraybackslash}X |}
  \hline
  Soggetto & E/R & Accessi & R/W \\
  \hline
  Unità operativa & E & 140 & R \\
  \hline
  Cura & R & 1400 & R \\
  \hline
  Composizione & R & 105 & R \\
  \hline
  Ospedale & E & 105 & R \\
  \hline
\end{tabularx}
\vspace{3pt}\newline
Costo operazione: 140R + 1400R + 105R + 105R = 1750 \newline Costo totale: 1750 * 200 (al giorno) = 350000 al giorno

\subsubsection*{18 - Aggiungere pazienti in cura presso un ospedale}
\textbf{Con ridondanza:}
\vspace{6pt}
\newline
\begin{tabularx}{\textwidth}{ 
  | >{\centering\arraybackslash}X 
  | >{\centering\arraybackslash}X 
  | >{\centering\arraybackslash}X 
  | >{\centering\arraybackslash}X |}
  \hline
  Soggetto & E/R & Accessi & R/W \\
  \hline
  Cura & R & 10 & R \\
  \hline
  Cura & R & 1 & W \\
  \hline
  Unità operativa & E & 1 & W \\
  \hline 
\end{tabularx}
\vspace{3pt}\newline
Costo operazione: 10R + 1W + 1W = 14 \newline Costo totale: 14 * 12500 (al giorno) = 175000 al giorno
\newline
\textbf{Senza ridondanza:}
\vspace{6pt}
\newline
\begin{tabularx}{\textwidth}{ 
  | >{\centering\arraybackslash}X 
  | >{\centering\arraybackslash}X 
  | >{\centering\arraybackslash}X 
  | >{\centering\arraybackslash}X |}
  \hline
  Soggetto & E/R & Accessi & R/W \\
  \hline
  Cura & R & 10 & R \\
  \hline
  Cura & R & 1 & W \\
  \hline 
\end{tabularx}
\vspace{3pt}\newline
Costo operazione: 10R + 1W = 12 \newline Costo totale: 12 * 12500 (al giorno) = 150000 al giorno

\subsubsection*{19 - Rimuovere un paziente in cura}
\textbf{Con ridondanza:}
\vspace{6pt}
\newline
\begin{tabularx}{\textwidth}{ 
  | >{\centering\arraybackslash}X 
  | >{\centering\arraybackslash}X 
  | >{\centering\arraybackslash}X 
  | >{\centering\arraybackslash}X |}
  \hline
  Soggetto & E/R & Accessi & R/W \\
  \hline
  Cura & R & 1 & W \\
  \hline
  Unità operativa & E & 1 & W \\
  \hline 
\end{tabularx}
\vspace{3pt}\newline
Costo operazione: 1W + 1W = 4 \newline Costo totale: 4 * 12500 (al giorno) = 50000 al giorno
\newline
\textbf{Senza ridondanza:}
\vspace{6pt}
\newline
\begin{tabularx}{\textwidth}{ 
  | >{\centering\arraybackslash}X 
  | >{\centering\arraybackslash}X 
  | >{\centering\arraybackslash}X 
  | >{\centering\arraybackslash}X |}
  \hline
  Soggetto & E/R & Accessi & R/W \\
  \hline
  Cura & R & 1 & W \\
  \hline 
\end{tabularx}
\vspace{3pt}\newline
Costo operazione: 1W = 2 \newline Costo totale: 2 * 12500 (al giorno) = 25000 al giorno
\vspace{3pt}
\newline
Totale con ridondanza: 42000 + 175000 + 150000 = 367000. 
\newline
Totale senza ridondanza: 350000 + 150000 + 25000 = 525000.
\vspace{3pt}
\newline
Risulta chiaro che la ridondanza riduce significativamente il carico sul sistema, quindi verrà mantenuto.

\subsection{Eliminazione delle gerarchie}
\subsubsection*{Referto}
La gerarchia di \emph{Referto} è stata eliminata con un collasso verso l'alto. Questo perché accedendo al referto si vuole accedere anche alle relative
informazioni contenute nelle entità figlie \emph{Intervento} e \emph{Visita}, e queste ultime non hanno ulteriori relazioni. Si è quindi deciso di accorpare gli attributi 
di entrambe le entità figlie nell'entità padre e aggiungere un attributo di tipo per semplificare lo schema.
\begin{figure}[H]
	\centering{}
	\includegraphics[width=\textwidth]{img/ger_referto_complete.png}
	\caption{Schema del collasso della gerarchia \emph{Referto}.}
	\label{img:ger_referto_complete}
\end{figure}

\subsubsection*{Persona}
La gerarchia di \emph{Persona} è totale ma è anche sovrapposta (un medico o un impiegato possono essere anche pazienti) per questo si è deciso di 
evitare un collasso verso l'alto che avrebbe richiesto 3 attributi booleani per il tipo e avrebbe reso difficile modellare i vincoli e le relazioni
delle entità figlie.

La copertura non è esclusiva quindi un collasso verso il basso avrebbe introdotto ridondanza, con il possibile problema di dati discordanti fra stesse 
entità in tabelle diverse, per questo si è deciso di evitare anche questa soluzione.

Seguendo le considerazioni precedenti si è deciso di trasformare la gerarchia sostituendola con associazioni, anche considerando che le entità figlie hanno
relazioni indipendenti dall'entità padre.
\begin{figure}[H]
	\centering{}
	\includegraphics[width=\textwidth]{img/ger_persona_complete.png}
	\caption{Schema del collasso della gerarchia \emph{Persona}.}
	\label{img:ger_persona_complete}
\end{figure}

\subsection{Eliminazione di attributi multivalore}
L'unico attributo multivalore presente nello schema è \emph{Telefono} nell'entità \emph{Persona}. Non essendo nota a priori la cardinalità massima
si decide di trasformare l'attributo in un'entità \emph{Telefono} a parte associata a \emph{Persona} tramite la relazione \emph{Utenza}.

\subsection{Scelta degli identificatori principali}
Nello schema E/R sono già evidenziate tutte le chiavi primarie di tutte le entità, segue una breve spiegazione su alcuni identificatori non banali:

La sala è identificata dal numero ma anche dall'ospedale in cui si trovano, questo perché i numeri delle sale sono univoci solo all'interno della
struttura in cui si trovano e potrebbero esistere sale con lo stesso numero in strutture diverse.

Anche per le unità operative è stato scelto un identificatore composto dal nome dell'unità e dalla struttura in cui si trova, perché potrebbero esistere
unità operative con stesso nome in strutture diverse.

Essendo l'inventario proprio di ogni ospedale anche i codici di inventario dell'attrezzatura sono in relazione all'ospedale in cui si trovano.

L'identificativo dell'appuntamento è inoltre composto dalla sala in cui si svolge, dalla data, e dall'ora. Questo impedisce che ci siano 2 appuntamenti che
iniziano nello stesso luogo nello stesso momento, ma non la sovrapposizione sfalsata di 2 appuntamenti che quindi andrà gestita diversamente.

\subsection{Eliminazione degli identificatori esterni}
\begin{itemize}
  \item Relazione \emph{Possiede} tra \emph{Attrezzatura} e \emph{Ospedale}: È stata eliminata con l'importazione di codice ospedale in attrezzatura.
  \item Relazione \emph{Struttura} tra \emph{Sala} e \emph{Ospedale}: È stata eliminata con l'importazione di codice ospedale in attrezzatura.
  \item Relazione \emph{Si svolge} tra \emph{Appuntamento} e \emph{Sala}: È stata eliminata con l'importazione di codice ospedale e numero in appuntamento.
  \item Relazione \emph{Composizione} tra \emph{Unità operativa} e \emph{Ospedale}: È stata eliminata con l'importazione di codice ospedale in unità operativa
\end{itemize}

\section{Traduzione delle entità e associazioni in relazioni}
\begin{itemize}
  \item \textbf{AMMINISTRATIVI}(\underline{Codice\_fiscale}, Ruolo, Codice\_ospedale) 
  \\ FK: Codice\_fiscale REFERENCES \emph{PERSONE}
  \\ FK: Codice\_ospedale REFERENCES \emph{OSPEDALI}
  \item \textbf{APPUNTAMENTI}(\underline{Codice\_ospedale}, \underline{Numero}, \underline{Data}, \underline{Orario}, Durata, Tipo, Paziente)
  \\ FK: Codice\_ospedale, Numero REFERENCES \emph{SALE}
  \\ FK: Paziente REFERENCES \emph{PAZIENTI}
  \item \textbf{ASL}(\underline{Codice}, Nome, Ind\_Citta, Ind\_Via, Ind\_Numero\_Civico)
  \item ATTREZZATURA(\underline{Codice\_ospedale}, \underline{Codice\_inventario}, Nome, Data\_manutenzione)
  \\ FK: Codice\_ospedale REFERENCES \emph{OSPEDALI}
  \item \textbf{COINVOLGIMENTI}(\underline{Referto}, \underline{Medico})
  \\ FK: Referto REFERENCES \emph{REFERTI}
  \\ FK: Medico REFERENCES \emph{PERSONALE\_SANITARIO}
  \item \textbf{CURE}(\underline{Paziente}, \underline{Codice\_ospedale}, \underline{Nome\_unita}, \underline{Data\_ingresso}, Data\_uscita*, Motivazione)
  \\ FK: Codice\_ospedale, Nome\_unita REFERENCES \emph{UNITA\_OPERATIVE}
  \\ FK: Paziente REFERENCES \emph{PAZIENTI}
  \item \textbf{LAVORA}(\underline{Codice\_ospedale}, \underline{Nome\_unita}, \underline{Codice\_fiscale})
  \\ FK: Codice\_ospedale, Nome\_unita REFERENCES \emph{UNITA\_OPERATIVE}
  \\ FK: Codice\_fiscale REFERENCES \emph{PERSONALE\_SANITARIO}
  \item \textbf{OSPEDALI}(\underline{Codice\_struttura}, Nome, Ind\_Citta, Ind\_Via, Ind\_Numero\_Civico, Cod\_ASL)
  \\ FK: Cod\_ASL REFERENCES \emph{ASL}
  \item \textbf{PAZIENTI}(\underline{Codice\_fiscale}, Data\_nascita, Cod\_ASL*)
  \\ FK: Codice\_fiscale REFERENCES \emph{PERSONE}
  \\ FK: Cod\_ASL REFERENCES \emph{ASL}
  \item \textbf{PERSONE}(\underline{Codice\_fiscale}, Nome, Cognome)
  \item \textbf{PERSONALE\_SANITARIO}(\underline{Codice\_fiscale}, Ruolo)
  \\ FK: Codice\_fiscale REFERENCES \emph{PERSONE}
  \item \textbf{PRESENZIA}(\underline{Medico}, \underline{Codice\_ospedale}, \underline{Numero\_sala}, \underline{Data}, \underline{Orario})
  \\ FK: Medico REFERENCES \emph{PERSONALE\_SANITARIO}
  \\ FK: Codice\_ospedale, Numero\_sala, Data, Orario REFERENCES \emph{APPUNTAMENTI}
  \item \textbf{REFERTI}(\underline{Codice\_referto}, Data\_emissione, Descrizione, Tipo, Terapia*, Procedura*, Esito*, Durata*, Codice\_ospedale, Paziente)
  \\ FK: Codice\_ospedale REFERENCES \emph{OSPEDALI}
  \\ FK: Paziente REFERENCES \emph{PAZIENTI}
  \item \textbf{SALE}(\underline{Codice\_ospedale}, \underline{Numero})
  \\ FK: Codice\_ospedale REFERENCES \emph{OSPEDALI}
  \item \textbf{TELEFONI}(\underline{Telefono}, \underline{Persona})
  \\ FK: Persona REFERENCES \emph{PERSONE}
  \item \textbf{UNITA\_OPERATIVE}(\underline{Codice\_ospedale}, \underline{Nome}, Capienza, Posti\_occupati)
  \\ FK: Codice\_ospedale REFERENCES \emph{OSPEDALI}
\end{itemize}

\begin{figure}[p]
  \begin{adjustbox}{addcode={\begin{minipage}{\width}}{\caption{%
    Schema logico finale.
    }\end{minipage}},rotate=270,center}
    \includegraphics[height=1.07\textwidth]{img/logic_final.png}
    \label{img:logic}
  \end{adjustbox}
\end{figure}

\appendix 
\chapter{Query SQL per la creazione del database}
\section{Creazione tabelle}

create database Hospital;\newline
use Hospital;\newline

\noindent create table AMMINISTRATIVI (

     Codice\_fiscale char(16) not null,

     Ruolo varchar(70) not null,

     Codice\_ospedale int not null,

     constraint FKPERS\_AMMINISTRATIVI\_ID primary key (Codice\_fiscale)); \newline

\noindent create table APPUNTAMENTI (

     Codice\_ospedale int not null,

     Numero\_sala int not null,

     Data\_ora timestamp not null,

     Durata int not null,

     Tipo varchar(80) not null,

     Paziente char(16) not null,

     constraint IDAPPUNTAMENTO\_ID primary key (Codice\_ospedale, Numero\_sala, Data\_ora)); \newline

\noindent create table ASL (

     Codice int not null AUTO\_INCREMENT,

     Nome varchar(30) not null,

     Ind\_Citta varchar(35) not null,

     Ind\_Via varchar(25) not null,

     Ind\_Numero\_civico char(8) not null,

     constraint IDASL primary key (Codice)); \newline

\noindent create table ATTREZZATURE (

     Codice\_ospedale int not null,

     Codice\_inventario int not null,

     Nome varchar(30) not null,

     Data\_manutenzione date not null,

     constraint IDATTREZZATURE primary key (Codice\_ospedale, Codice\_inventario)); \newline

\noindent create table COINVOLGIMENTI (

     Referto int not null,

     Medico char(16) not null,

     constraint IDCOINVOLGIMENTI primary key (Referto, Medico)); \newline

\noindent create table CURE (

     Paziente char(16) not null,

     Codice\_ospedale int not null,

     Nome\_unita varchar(30) not null,

     Data\_ingresso date not null,

     Data\_uscita date,

     Motivazione varchar(400) not null,

     constraint IDCURE primary key (Paziente, Codice\_ospedale, Nome\_unita, Data\_ingresso)); \newline

\noindent create table LAVORA (

     Codice\_ospedale int not null,

     Nome\_unita varchar(30) not null,

     Codice\_fiscale char(16) not null,

     constraint IDLAVORA primary key (Codice\_fiscale, Codice\_ospedale, Nome\_unita)); \newline

\noindent create table OSPEDALI (

     Codice\_struttura int not null AUTO\_INCREMENT,

     Nome varchar(60) not null,

     Ind\_Citta varchar(35) not null,

     Ind\_Via varchar(25) not null,

     Ind\_Numero\_civico char(8) not null,

     Cod\_ASL int not null,

     constraint IDOSPEDALI primary key (Codice\_struttura)); \newline

\noindent create table PAZIENTI (

     Codice\_fiscale char(16) not null,

     Data\_nascita date not null,

     Cod\_ASL int,

     constraint FKPERS\_PAZIENTI\_ID primary key (Codice\_fiscale)); \newline

\noindent create table PERSONE (

     Nome varchar(20) not null,

     Cognome varchar(20) not null,

     Codice\_fiscale char(16) not null,

     constraint IDPERSONE\_ID primary key (Codice\_fiscale)); \newline

\noindent create table PERSONALE\_SANITARIO (

     Codice\_fiscale char(16) not null,

     Ruolo varchar(30) not null,

     constraint FKPERS\_SANITARI\_ID primary key (Codice\_fiscale)); \newline

\noindent create table PRESENZIA (

     Medico char(16) not null,

     Codice\_ospedale int not null,

     Numero\_sala int not null,

     Data\_ora timestamp not null,

     constraint IDPRESENZIA primary key (Medico, Codice\_ospedale, Numero\_sala, Data\_ora)); \newline

\noindent create table REFERTI (

     Codice\_referto int not null AUTO\_INCREMENT,

     Data\_emissione date not null,

     Descrizione varchar(1000) not null,

     Tipo varchar(10) not null,

     Terapia varchar(500),

     Procedura varchar(500),

     Esito varchar(500),

     Durata int,

     Codice\_ospedale int not null,

     Paziente char(16) not null,

     constraint IDREFERTI\_ID primary key (Codice\_referto)); \newline

\noindent create table SALE (

     Codice\_ospedale int not null,

     Numero int not null,

     constraint IDSALE primary key (Codice\_ospedale, Numero)); \newline

\noindent create table TELEFONI (

     Telefono char(15) not null,

     Persona char(16) not null,

     constraint IDTELEFONI primary key (Telefono, Persona)); \newline

\noindent create table UNITA\_OPERATIVE (

     Codice\_ospedale int not null,

     Nome varchar(30) not null,

     Capienza int not null,

     Posti\_occupati int not null,

     constraint IDUNITA\_OPERATIVE primary key (Codice\_ospedale, Nome)); \newline

\section{Constraints}

\noindent alter table AMMINISTRATIVI add constraint FKPERS\_AMMINISTRATIVI\_FK

     foreign key (Codice\_fiscale)

     references PERSONE (Codice\_fiscale)

     ON DELETE CASCADE; \newline

\noindent alter table AMMINISTRATIVI add constraint FKIMPIEGATI

     foreign key (Codice\_ospedale)

     references OSPEDALI (Codice\_struttura)

     ON DELETE CASCADE; \newline

\noindent alter table APPUNTAMENTI add constraint FKSI\_SVOLGE

     foreign key (Codice\_ospedale, Numero\_sala)

     references SALE (Codice\_ospedale, Numero); \newline

\noindent alter table APPUNTAMENTI add constraint FKPRENOTA

     foreign key (Paziente)

     references PAZIENTI (Codice\_fiscale)

     ON DELETE CASCADE; \newline

\noindent alter table ATTREZZATURE add constraint FKPOSSIEDE

     foreign key (Codice\_ospedale)

     references OSPEDALI (Codice\_struttura)

     ON DELETE CASCADE; \newline

\noindent alter table COINVOLGIMENTI add constraint FKCOI\_PER

     foreign key (Medico)

     references PERSONALE\_SANITARIO (Codice\_fiscale); \newline

\noindent alter table COINVOLGIMENTI add constraint FKCOI\_REF

     foreign key (Referto)

     references REFERTI (Codice\_referto)

     ON DELETE CASCADE; \newline

\noindent alter table CURE add constraint FKCura\_Unita\_operativa

     foreign key (Codice\_ospedale, Nome\_unita)

     references UNITA\_OPERATIVE (Codice\_ospedale, Nome); \newline

\noindent alter table CURE add constraint FKPaziente

     foreign key (Paziente)

     references PAZIENTI (Codice\_fiscale); \newline

\noindent alter table LAVORA add constraint FKPersonale

     foreign key (Codice\_fiscale)

     references PERSONALE\_SANITARIO (Codice\_fiscale)

     ON DELETE CASCADE; \newline

\noindent alter table LAVORA add constraint FKLavora\_Unita\_operativa

     foreign key (Codice\_ospedale, Nome\_unita)

     references UNITA\_OPERATIVE (Codice\_ospedale, Nome)

     ON DELETE CASCADE; \newline

\noindent alter table OSPEDALI add constraint FKAPPARTENENZA

     foreign key (Cod\_ASL)

     references ASL (Codice); \newline

\noindent alter table PAZIENTI add constraint FKREGISTRAZIONE

     foreign key (Cod\_ASL)

     references ASL (Codice)

     ON DELETE SET NULL; \newline

\noindent alter table PAZIENTI add constraint FKPERS\_PAZIENTI\_FK

     foreign key (Codice\_fiscale)

     references PERSONE (Codice\_fiscale)

     ON DELETE CASCADE; \newline 

\noindent alter table PERSONALE\_SANITARIO add constraint FKPERS\_SANITARI\_FK

     foreign key (Codice\_fiscale)

     references PERSONE (Codice\_fiscale)

     ON DELETE CASCADE; \newline

\noindent alter table PRESENZIA add constraint FKPRE\_APP

     foreign key (Codice\_ospedale, Numero\_sala, Data\_ora)

     references APPUNTAMENTI (Codice\_ospedale, Numero\_sala, Data\_ora)

     ON DELETE CASCADE; \newline

\noindent alter table PRESENZIA add constraint FKPRE\_PER

     foreign key (Medico)

     references PERSONALE\_SANITARIO (Codice\_fiscale); \newline

\noindent alter table REFERTI add constraint FKEMISSIONE

     foreign key (Codice\_ospedale)

     references OSPEDALI (Codice\_struttura)

     ON DELETE CASCADE; \newline

\noindent alter table REFERTI add constraint FKASSOCIAZIONE

     foreign key (Paziente)

     references PAZIENTI (Codice\_fiscale)

     ON DELETE CASCADE; \newline

\noindent alter table SALE add constraint FKSTRUTTURA

     foreign key (Codice\_ospedale)

     references OSPEDALI (Codice\_struttura)

     ON DELETE CASCADE; \newline

\noindent alter table TELEFONI add constraint FKUTENZA

     foreign key (Persona)

     references PERSONE (Codice\_fiscale)

     ON DELETE CASCADE; \newline

\noindent alter table UNITA\_OPERATIVE add constraint FKCOMPOSIZIONE

     foreign key (Codice\_ospedale)

     references OSPEDALI (Codice\_struttura)

     ON DELETE CASCADE; \newline

\chapter{Query SQL delle operazioni principali}

\begin{enumerate}[leftmargin=0cm,itemindent=.5cm,labelwidth=\itemindent,labelsep=0cm,align=left]
    \item \textbf{Aggiungere un nuovo ospedale:} \newline
    INSERT INTO OSPEDALI(Nome, Ind\_Citta, Ind\_Via, Ind\_Numero\_Civico, Cod\_ASL) 

    VALUES(?, ?, ?, ?, ?)

    \item \textbf{Aggiungere una nuova unità operativa:} \newline
    INSERT INTO UNITA\_OPERATIVE(Codice\_ospedale, Nome, Capienza, Posti\_occupati)

    VALUES(?, ?, ?, ?)

    \item \textbf{Rimuovere una unità operativa:} \newline
    DELETE FROM UNITA\_OPERATIVE

    WHERE Codice\_ospedale = ? AND Nome = ?

    \item \textbf{Aggiungere un nuovo paziente:} 
    \begin{itemize}
        \item \textbf{Controlliamo se il paziente che vogliamo aggiungere è già in PERSONE} \newline
        SELECT COUNT(*)

        FROM PERSONE

        WHERE Codice\_fiscale = ?

        \item \textbf{Se questa query restituisce un valore uguale a 0 allora procediamo ad aggiungere la persona, 
        altrimenti saltiamo al prossimo punto:} \newline
        INSERT INTO PERSONE(Nome, Cognome, Codice\_fiscale)

        VALUES(?, ?, ?)

        \item \textbf{Dopodichè si procede con la registrazione del paziente:} \newline
        INSERT INTO PAZIENTI(Codice\_fiscale, Data\_nascita, Cod\_ASL)

        VALUES(?, ?, ?)

    \end{itemize}
    \item \textbf{Aggiungere un impiegato o medico:} 
    \begin{itemize}
        \item \textbf{Controlliamo se l'impiegato/medico che vogliamo aggiungere è già in PERSONE} \newline
        SELECT COUNT(*)

        FROM PERSONE

        WHERE Codice\_fiscale = ?

        \item \textbf{Se questa query restituisce un valore uguale a 0 allora procediamo ad aggiungere la persona, 
        altrimenti saltiamo al prossimo punto:} \newline
        INSERT INTO PERSONE(Nome, Cognome, Codice\_fiscale)

        VALUES(?, ?, ?)

        \item \textbf{Dopodichè si aggiunge ad Amministrativi:} \newline
        INSERT INTO AMMINISTRATIVI(Codice\_fiscale, Ruolo, Codice\_ospedale)

        VALUES(?, ?, ?)

        \item \textbf{Oppure al Personale Sanitario:} \newline
        INSERT INTO PERSONALE\_SANITARIO(Codice\_fiscale, Ruolo)

        VALUES(?, ?)

    \end{itemize}
    \item \textbf{Fissare un appuntamento:} 
    \begin{itemize}
        \item \textbf{Cerco appuntamenti già fissati con lo stesso medico o nella stessa sala che si sovrapporrebbero a quello che si vorrebbe fissare.
        I parametri della seguente query sono 4, nell'ordine la data e ora, la durata, il medico, e la sala del nuovo appuntamento che si vuole fissare.} \newline
        
        SET @newdate = ?;

        SET @durata = ?;

        SELECT *

        FROM APPUNTAMENTI

        INNER JOIN PRESENZIA

        ON APPUNTAMENTI.Numero\_sala = PRESENZIA.Numero\_sala AND APPUNTAMENTI.Codice\_ospedale = PRESENZIA.Codice\_ospedale
        
        WHERE (Medico = ? AND

            (TIMESTAMPDIFF(SECOND, @newdate , PRESENZIA.Data\_ora) \textless= 0 AND TIMESTAMPDIFF(SECOND, TIMESTAMPADD(MINUTE, Durata, PRESENZIA.Data\_ora), @newdate) \textless 0) OR
            
            (TIMESTAMPDIFF(SECOND, TIMESTAMPADD(MINUTE, @durata, @newdate), PRESENZIA.Data\_ora) \textless 0 AND TIMESTAMPDIFF(SECOND, @newdate, PRESENZIA.Data\_ora) \textgreater 0))
            
            OR
            
            (PRESENZIA.Numero\_sala = ? AND PRESENZIA.Codice\_ospedale = ? AND
            
            (TIMESTAMPDIFF(SECOND, @newdate, PRESENZIA.Data\_ora) \textless= 0 AND TIMESTAMPDIFF(SECOND, TIMESTAMPADD(MINUTE, Durata, PRESENZIA.Data\_ora), @newdate) \textless 0) OR
            
            (TIMESTAMPDIFF(SECOND, TIMESTAMPADD(MINUTE, @durata, @newdate), PRESENZIA.Data\_ora) \textless 0 AND TIMESTAMPDIFF(SECOND, @newdate, PRESENZIA.Data\_ora) \textgreater 0))

        \item \textbf{Se la query restituisce un numero vuol dire che l'appuntamento si sovrapporrebbe ad un altro già fissato, altrimenti si può procedere a inserirlo 
        con la seguente query:} \newline
        
        INSERT INTO APPUNTAMENTI(Codice\_ospedale, Numero\_sala, Data\_ora, Durata, Tipo, Paziente)
        
        VALUES(?, ?, ?, ?, ?, ?)

        INSERT INTO PRESENZIA(Medico, Codice\_ospedale, Numero\_sala, Data\_ora)
        
        VALUES(?, ?, ?, ?)

    \end{itemize}
    \item \textbf{Cancellare un appuntamento:} Per cancellare un appuntamento è sufficiente eliminarlo dalla tabella appuntamenti, i vincoli si occuperanno di eliminare a cascata
    le presenze dei medici all'appuntamento dalla tabella presenzia.

    DELETE FROM APPUNTAMENTI
    
    WHERE Codice\_ospedale = ? AND Numero\_sala = ? AND Data\_ora = ?

    \item \textbf{Aggiungere un nuovo referto:} \newline
    
    INSERT INTO REFERTI(Data\_emissione, Descrizione, Tipo, Terapia, Procedura, Esito, Durata, Codice\_ospedale, Paziente)
    
    VALUES(?, ?, ?, ?, ?, ?, ?, ?, ?)

    \item \textbf{Ricercare referti per medico o per paziente}
    \begin{itemize}
        \item Per avere tutti i referti associati ad un paziente specifico si utilizza la seguente query:
        
        SELECT *
        
        FROM REFERTI
        
        WHERE Paziente = ?

        \item Per avere i referti associati ad un medico invece si usa la seguente:
        
        SELECT *
       
        FROM REFERTI
       
        WHERE Codice\_referto IN (

            SELECT Referto
            
            FROM COINVOLGIMENTI
            
            WHERE Medico = ?
            
            )
    \end{itemize}
    
    \item \textbf{Aggiungere nuova attrezzatura:} \newline
    
    INSERT INTO ATTREZZATURE(Codice\_ospedale, Codice\_inventario, Nome, Data\_manutenzione)
    
    VALUES(?, ?, ?, ?)

    \item \textbf{Rimuovere attrezzatura:} \newline
    
    DELETE FROM ATTREZZATURE

    WHERE Codice\_ospedale = ? AND Codice\_inventario = ?

    \item \textbf{Aggiornare la data di manutenzione di un'attrezzatura} \newline
    UPDATE ATTREZZATURE
    
    SET Data\_emissione = ?
    
    WHERE Codice\_ospedale = ? AND Codice\_inventario = ?

    \item \textbf{Ricercare ospedali con determinate unità operative} \newline
    SELECT *

    FROM OSPEDALI
    
    WHERE Codice\_struttura IN (
    
        SELECT Codice\_ospedale
    
        FROM UNITA\_OPERATIVE
    
        WHERE Nome = ?

    )

    \item \textbf{Ricerca di ospedali appartenenti ad un'ASL specifica} \newline
    SELECT *
    
    FROM OSPEDALI
    
    WHERE Cod\_ASL = ?

    \item \textbf{Aggiungere una nuova sala ad un ospedale} \newline
    INSERT INTO SALE(Codice\_ospedale, Numero)
    
    VALUES(?, ?)

    \item \textbf{Rimuovere una sala da un ospedale} \newline
    DELETE FROM SALE
    
    WHERE Codice\_ospedale = ?, Numero = ?

    \item \textbf{Ricercare ospedali con posti liberi in una determinata unità operativa} \newline
    SELECT *
    
    FROM OSPEDALI
    
    WHERE Codice\_struttura IN (
    
        SELECT Codice\_ospedale
        
        FROM UNITA\_OPERATIVE
        
        WHERE Nome = ? AND Capienza \textgreater Posti\_occupati 
    
    )

    \item \textbf{Aggiungere pazienti in cura presso un ospedale} \newline
    INSERT INTO CURE(Paziente, Codice\_ospedale, Nome\_unita, Data\_ingresso, Data\_uscita, Motivazione)
    
    VALUES(?, ?, ?, ?, ?, ?)

    \item \textbf{Rimuovere un paziente in cura presso un ospedale} \newline
    UPDATE CURE

    SET Data\_uscita = ?

    WHERE Paziente = ?, Codice\_ospedale = ?, Nome\_unita = ?

\end{enumerate}

\chapter{Guida utente all'applicativo}

L'applicativo realizzato consente di interagire con il database per gli aspetti di inserimento, cancellazione, modifica e consultazione dei dati.
All'avvio vengono richieste le credenziali per accedere all'istanza di mySQL in esecuzione sul computer locale. Inoltre se non viene trovato il database l'applicazione chiede se lo si 
vuole creare e nel caso procede all'esecuzione delle query necessarie.
La schermata principale consente di scegliere tra la sezione di modifica e quella di consultazione.
In ognuna delle sezioni di modifica è possibile scegliere tra le seguenti operazioni:
\begin{itemize}
    \item Inserimento: È sufficiente compilare i campi e premere il pulsante "Inserisci" per eseguire l'inserimento.
    \item Cancellazione: Selezionare un entità da eliminare con l'apposito tasto "Seleziona" e premere il pulsante "Rimuovi" per eseguire la cancellazione.
    \item Modifica (non sempre disponibile): Selezionare un entità da modificare con l'apposito tasto "Seleziona", modificarne i campi (i campi modificabili sono evidenziati in blu),
    dopodichè premere il pulsante "Aggiorna" per eseguire la modifica.
\end{itemize}
Un messaggio a schermo indicherà se l'operazione è andata a buon fine.

\begin{figure}[H]
	\centering{}
	\includegraphics[width=\textwidth]{img/Screenshot_aggiunta_referti.png}
	\caption{Schermata di aggiunta dei referti.}
	\label{img:add_referti}
\end{figure}

\begin{figure}[H]
	\centering{}
	\includegraphics[width=\textwidth]{img/Screenshot_aggiunta_ricoveri.png}
	\caption{Schermata di aggiunta dei referti.}
	\label{img:add_ricoveri}
\end{figure}

Nella sezione di consultazione è invece possibile visualizzare i dati divisi per categorie e ricercarli attraverso filtri propri di ogni entità che si vuole visualizzare.

\begin{figure}[H]
	\centering{}
	\includegraphics[width=\textwidth]{img/Screenshot_ricerca_ospedale.png}
	\caption{Schermata di ricerca degli ospedali.}
	\label{img:search_hospital}
\end{figure}

\end{document}
