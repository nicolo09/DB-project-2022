\documentclass[a4paper,12pt]{report}

\usepackage{alltt, fancyvrb, url}
\usepackage{graphicx}
\usepackage[utf8]{inputenc}
\usepackage{float}
\usepackage{hyperref}

% Questo commentalo se vuoi scrivere in inglese.
\usepackage[italian]{babel}

\usepackage[italian]{cleveref}

\title{Relazione per\\``Basi di dati''}

\author{Nicolò Guerra \and
Filippo Casadei}

\begin{document}

\maketitle

\tableofcontents

\chapter{Analisi dei requisiti}

Si vuole realizzare un database per la gestione di sistemi ospedalieri. Il sistema dovrà immagazzinare dati relativi a diverse ASL, 
agli ospedali ad esse associate, pazienti, medici. Dovrà registrare inoltre referti relativi a visite, interventi e appuntamenti.

\section{Intervista}
Si vuole tenere traccia dei referti prodotti nei vari ospedali delle varie ASL della regione. Un referto può essere prodotto da un
intervento o da una visita/esame. Un referto è prodotto da un medico in un ospedale, ed è riferito ad un paziente. Dei pazienti si
vuole memorizzare nome, cognome, codice fiscale, data di nascita, ASL di appartenenza e referti associati.
Di un impiegato si vuole memorizzare nome, cognome, codice fiscale e ruolo. Di un medico si vuole inoltre tenere traccia dello 
storico degli interventi.
Un referto viene prodotto in una specifica data e nel caso sia di un intervento, si vuole sapere la procedura, l'esito, la durata e i medici 
coinvolti. Nel caso sia di una visita invece si vuole avere una breve descrizione di quest'ultima e l'eventuale terapia prescritta.
Si vogliono memorizzare inoltre i seguenti dati per gli ospedali: nome, indirizzo, ASL di appartenenza, posti disponibili e persone 
che lavorano in una struttura, e se questo dispone di un pronto soccorso.
Un ospedale si compone inoltre di più unità operative, ognuna caratterizzata da un nome, un dirigente, e dal personale che vi lavora.
Ogni ospedale ha delle attrezzature, identificate da un codice di inventario e di cui si vuole memorizzare il nome e la data dell'ultima 
manutenzione.
Si vogliono infine memorizzare degli appuntamenti, che possono essere ad esempio prenotazioni per interventi e coinvolgono uno o più 
pazienti e uno o più medici in una certa sala dell'ospedale ad una precisa data e ora. Un medico o un paziente non possono avere più 
appuntamenti nello stesso momento. Una sala non può essere occupata da 2 appuntamenti contemporaneamente.
\section{Concetti principali da modellare}
\begin{itemize}
  \item ASL: Azienda sanitaria locale, gestisce la sanità di una singola zona di competenza, solitamente provinciale
  \item Ospedale: Struttura in cui avvengono visite e interventi dei pazienti
  \item Referto: Documento prodotto da un medico come resoconto di una visita o un intervento
  \item Paziente: Persona sottoposta a trattamenti ospedalieri
  \item Medico: Persona che lavora in un ospedale (o più di uno) e effettua trattamenti
  \item Impiegato: Persona che lavora in un ospedale ma non effettua trattamenti e non può firmare referti
  \item Unità operativa: Reparto dell'ospedale specializzato in determinati tipi di trattamenti
  \item Attrezzatura: Macchinario utilizzato per particolari interventi e/o visite
  \item Appuntamento: Programmazione di visita o intervento
  \item Visita: Consulenza con un medico riguardante lo stato di salute del paziente
  \item Intervento: Operazione ad un paziente da parte di uno o più medici
  \item Sala: Stanza in cui avvengono visite e interventi
\end{itemize}
\section{Riscrittura dell'intervista corretta}
\end{document}